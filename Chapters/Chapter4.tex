% Chapter 4
 
\chapter{Preprocessing} % Main chapter title

\label{Preprocessing} % For referencing the chapter elsewhere, use \ref{Chapter1} 

%----------------------------------------------------------------------------------------

% Define some commands to keep the formatting separated from the content 

%----------------------------------------------------------------------------------------

\section{Technique And Design}
The universal database contains raw data of academic performance of the students.There are plenty of problems in this Database which has been discussed in \textit{Chapter3}.A rigorous preprocessing technique is highly needed for turn the database into a suitable one for the classification implementation.The classification algorithm demands a clean and quality dataset without which it can't proceed efficiently.In this procedure,there were several steps taken for the data preprocessing. 
\subsection{Data Cleaning}

 One of the main problems with the universal database was,there are several attributes where no value was not inserted at all.In result,there are plenty of blank in the data sheet. As the first step of cleaning,these missing values were filled up using intuition and global constant. For example, data sheet 1 had some blank attributes like the Table \ref{tab:tit0}.
\begin {table}[H]
\caption {Blank attributes in the Universal Database} \label{tab:tit0}
\begin{center}
\begin{tabular}{ | m{1cm} | m{2cm}| m{0.5cm}| m{2cm} | m{2cm} | m{2cm} |  m{0.5cm} | } 
\hline
Serial & Department & ... & CourseName & PartAMark & PartBMark & ... \\ 
\hline
65 & CSE & ... & CSE100 & - & - & ... \\ 
\hline
65 & CSE & ... & CSE210 & - & - & ... \\ 
\hline
... & ... & ... & ... & ... & ... & ... \\ 
\hline
\end{tabular}
\end{center}
\end{table}
The \textit{PartAMark \& PartBMark} fields are blank because the tuples are for sessional courses and only theory courses have Part A and Part B factor. \newline
In this case,these two fields have been filled up with 0.After the process,table looked like Table \ref{tab:title0}.
\begin {table}[H]
\caption {Universal database after filling up blank attributes} \label{tab:title0}
\begin{center}
\begin{tabular}{ | m{1cm} | m{2cm}| m{0.5cm}| m{2cm} | m{2cm} | m{2cm} |  m{0.5cm} | } 
\hline
Serial & Department & ... & CourseName & PartAMark & PartBMark & ... \\ 
\hline
65 & CSE & ... & CSE100 & 0 & 0 & ... \\ 
\hline
65 & CSE & ... & CSE210 & 0 & 0 & ... \\ 
\hline
... & ... & ... & ... & ... & ... & ... \\ 
\hline
\end{tabular}
\end{center}
\end{table}
\\There are several attributes in the dataset which are not really needed for the classification implementation. As we are going to implement the excellency of a student in academic career,attributes like \textit{District,Thana,} \textit{PlaceOfBirth,AdmissionDate,StartingDate} are not going to affect the classification algorithm anyway. So, all the database has been got rid of all the redundant attributes. Before this procedure,the database was like Table \ref{tab:ti1}.
\begin {table}[H]
\caption {Redundant attributes in Database} \label{tab:ti1}
\begin{center}
\begin{tabular}{ | m{1cm} | m{2cm}| m{1.5cm}| m{1.5cm} | m{2cm} | m{2cm} |  m{0.5cm} | } 
\hline
Serial & Department & District & Thana & CourseName & LetterGrade & ... \\ 
\hline
4478 & NAME & Dhaka & Dhamrai & MATH141 & A & ... \\ 
\hline
14251 & MME & Satkhira & koloroa & PHY143 & B+ & ... \\ 
\hline
... & ... & ... & ... & ... & ... & ... \\ 
\hline
\end{tabular}
\end{center}
\end{table}
After data cleaning process,the redundant attributes have been taken care of. Table \ref{tab:ti2} is the outcome. 
\begin {table}[!H]
\caption {Database after cleaning of redundant attributes} \label{tab:ti2}
\begin{center}
\begin{tabular}{ | m{1cm} |  m{2cm} | m{2cm} | m{2cm} |  m{0.5cm} | } 
\hline
Serial & Department &  CourseName & LetterGrade & ... \\ 
\hline
4478 & NAME & MATH141 & A & ... \\ 
\hline
14251 & MME & PHY143 & B+ & ... \\ 
\hline
... & ... & ... & ... & ... \\ 
\hline
\end{tabular}
\end{center}
\end{table}
\\ The universal database is filled with plenty of inconsistent data.These data types are not compatible with the procedure of ID3 classification. Some steps were taken to change these inconsistent data type to a consistent one. \newline
The \textit{Gender} and \textit{HallStatus} fields are filled with binary value.\textit{Gender} is represented as \textit{Male} or \textit{Female} and \textit{HallStatus} is represented with \textit{Resident} or \textit{Attached}.
\begin {table}[!H]
\caption {Database wth binary values} \label{tab:title}
\begin{center}
\begin{tabular}{ | m{1cm} |  m{2cm} | m{2cm} | m{2cm} |  m{0.5cm} | } 
\hline
Serial & Department &  Gender & HallStatus & ... \\ 
\hline
4478 & NAME & Male & Attached & ... \\ 
\hline
14251 & MME & Male & Resident & ... \\ 
\hline
... & ... & ... & ... & ... \\ 
\hline
\end{tabular}
\end{center}
\end{table}
The values are converted to numeric  from binary so that the data set becomes compatible with the ID3 procedure.
\begin {table}[H]
\caption {Database after conversion of binary attributes to numeric value} \label{tab:title}
\begin{center}
\begin{tabular}{ | m{1cm} |  m{2cm} | m{2cm} | m{2cm} |  m{0.5cm} | } 
\hline
Serial & Department &  Gender & HallStatus & ... \\ 
\hline
4478 & NAME & 0 & 1 & ... \\ 
\hline
14251 & MME & 0 & 0 & ... \\ 
\hline
... & ... & ... & ... & ... \\ 
\hline
\end{tabular}
\end{center}
\end{table}
\\ A crucial fault in the universal database is,it doesn't have any \textit{CourseCreditHour} attribute and some data sheets don't have the correct value of \textit{CGPA} as well.The \textit{CGPA} field is rounded in some data sheets.Which does not represent the accurate value.But for proper calculation,accurate \textit{CGPA} and \textit{GPA} is must.
\begin {table}[!H]
\caption {Database before Data cleaning} \label{tab:title}
\begin{center}
\begin{tabular}{ | m{1cm} |  m{2cm} | m{2.1cm} | m{2cm} | m{2cm} | m{0.5cm} | } 
\hline
Serial & Department &  CourseName & GPA & CGPA & ... \\ 
\hline
4478 & NAME & MATH141 & 3 & 3 & ... \\ 
\hline
14251 & MME & PHY143 & 3 & 3 & ... \\ 
\hline
... & ... & ... & ... & ... & ... \\ 
\hline
\end{tabular}
\end{center}
\end{table}
In this case,with the help of other attributes of the database, \textit{CourseCreditHour}, \textit{GPA} and \textit{CGPA} is calculated.\newline
At first, \textit{Numberpercentage} is calculated from \textit{LetterGrade} attribute.
\begin {table}[!H]
\caption {Letter Grade and it's respective percentage of number} \label{tab:title}
\begin{center}
\begin{tabular}{ | m{2.2cm} |  m{4cm} | } 
\hline
LetterGrade & NumberPercentage \\ 
\hline
A+ & 80\%-100\% \\ 
\hline
A & 75\%-79\% \\
\hline
A- & 70\%-74\% \\
\hline
B+ & 65\%-69\% \\
\hline
B & 60\%-64\% \\
\hline
B- & 55\%-59\% \\
\hline
C+ & 50\%-54\% \\
\hline
C & 45\%-49\% \\ 
\hline
D & 40\%-44\% \\
\hline
\end{tabular}
\end{center}
\end{table}
Then,the \textit{CourseCreditHour} is calculated from the following equation.
\begin{equation} \label{eq1}
\begin{split}
CourseCreditHour & = \frac{TotalNumber}{NumberPercentage} \\
\end{split}
\end{equation}
After the modification in \textit{CourseCreditHour},the database looks like Table \ref{tab:tit}.
\begin {table}[ht]
\caption {Database after addition of CourseCreditHour attribute} \label{tab:tit}
\begin{center}
\begin{tabular}{ | m{1cm} |  m{2cm} | m{2.1cm} | m{1cm} | m{3.1cm} | m{0.5cm} | } 
\hline
Serial & Department &  CourseName & GPA & CourseCreditHour & ... \\ 
\hline
4478 & NAME & MATH141 & 3.5 & 4 & ... \\ 
\hline
14251 & MME & PHY143 & 3.25 & 3 & ... \\ 
\hline
... & ... & ... & ... & ... & ... \\ 
\hline
\end{tabular}
\end{center}
\end{table}

In the ID3 implementation,the final \textit{CGPA} of student is used only.It is calculated using the 
\textit{CourseCreditHour} and \textit{GPA}  attribute of all courses the student is enrolled to.
 \begin{equation} \label{eq1}
\begin{split}
CGPA & = \frac{\sum (CourseCreditHour * CourseGPA)}{\sum CourseCreditHour} \\
\end{split}
\end{equation}
In this phase,one student has only one tuple in the database.


\subsection{Data Normalization}
We used normalization by decimal scaling normalizes by moving the decimal point of values of attribute A. The
number of decimal points moved depends on the maximum absolute value of A. A value, $v_i$, of A is normalized
to $v^{'}_i$ by computing $$ v^{'}_i = \dfrac{v_i}{10^{j}} $$
where j is the smallest integer such that $ max(|v^{'}_i|) \textless 1$.

For example, we normalized class attendance mark by calculating marks for each credit. Maximum 10 marks are given for a theory course. So, we normalized the data by calculating attendance for each credit than dividing it by the total theory credit completed by the student.
\newpage
\subsection{Data Transformation}
We used numeric data format for our algorithm. So we converted string values to a corresponding integer or numeric values.
For example: We gave each department unique numeric id instead of using their name code. 'EEE' is replaced bt 1,'CSE'
is replaced by 2 so on.
Similarly for 'Hall Status' we replaced 'Resident' with 1 and 'Attached' with 0. 
Similarly for 'Gender'  female is indicated by 0 and male is indicated by 1.
%----------------------------------------------------------------------------------------

%----------------------------------------------------------------------------------------

\section{Results}

After implementing some crucial data preprocessing procedures,the modified database is at last suitable for the next step,implementation of ID3 algorithm using decision tree.The final test data has only the relevant attributes in compatible format.The attributes are :
\begin{itemize}
\item Serial
\item Department
\item Hall Status
\item Gender
\item ClassTestMark
\item AttendanceMark
\item CGPA
\item  TotalCourseCompleted
\end{itemize} 